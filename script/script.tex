\documentclass{article}
\usepackage[utf8]{inputenc}
\usepackage[american]{babel}
\usepackage{hyphenat}
\usepackage[activate={true,nocompatibility}, final, tracking=true, kerning=true, factor=1100, stretch=10, shrink=10]{microtype}
% Prevent hyphenization
\tolerance=1
\emergencystretch=\maxdimen
\hyphenpenalty=10000
\hbadness=10000

% Math
\newcommand\hmmax{0}
\newcommand\bmmax{0}
\usepackage{bm}
\usepackage{amsmath, amsfonts, amssymb, mathrsfs,extarrows}
\usepackage{commath}
\usepackage[retainorgcmds]{IEEEtrantools}
\usepackage{siunitx}
\usepackage{multirow}
\usepackage[widespace]{fourier}

% Tables
\usepackage{array}
\usepackage{caption}
\usepackage[figuresright]{rotating}
\usepackage{multirow}
\usepackage{makecell}
\usepackage{footnote}
\makesavenoteenv{tabular}

% Algorithm
\usepackage[section]{algorithm}
\usepackage{algpseudocode}

\usepackage[hmarginratio=1:1,textwidth=360pt,textheight=595.8pt]{geometry}
% Packages
\usepackage{import}

% Control structures
\usepackage{ifthen}

% Listing
\usepackage{listings}
\usepackage{xcolor}
\usepackage{float}

% Review notes:
\usepackage{xargs}
\usepackage[textwidth=30mm,textsize=footnotesize]{todonotes} % to create comments (useful to your advisor!)
\newcommandx{\bpf}[2][1=]{\todo[linecolor=blue,backgroundcolor=blue!25,bordercolor=blue,#1]{#2}} % Bernardo notes

% Equation color background
\usepackage{mdframed}

\newmdenv[
    hidealllines=true,
    backgroundcolor=black!20,
    skipbelow=\baselineskip,
    skipabove=\baselineskip
]{highlight}

\newcounter{problem}[section]\setcounter{problem}{1}
\renewcommand{\theproblem}{\arabic{section}.\arabic{problem}}
\newenvironment{problem}[2][]{%
    \refstepcounter{problem}

    \mdfsetup{hidealllines=true,
    backgroundcolor=black!20,
    skipbelow=\baselineskip,
    skipabove=\baselineskip,
    frametitle={Problem~\theproblem~|~#1}}

\begin{mdframed}[]\relax}{%
\end{mdframed}}

% Enumeration
\usepackage{enumerate}


% Images
\usepackage[labelformat=simple]{subcaption}
\renewcommand\thesubfigure{(\alph{subfigure})}
\renewcommand\thesubtable{(\alph{subtable})}
\usepackage{graphicx}
\graphicspath{ {figures/} }
\usepackage{array}
\usepackage[section]{placeins}
\usepackage{color}
% \usepackage{subcaption}
% \usepackage{subfig}



% Images SVG
\usepackage{import}
\usepackage{xifthen}
\usepackage{pdfpages}
\usepackage{transparent}

\newcommand{\incfig}[1]{
    \def\svgwidth{0.3\columnwidth}
    \import{images/studies/minkowski/fundamental_forms_2D/}{#1.pdf_tex}}


% Floating environment for listings
\floatstyle{plain}
\newfloat{lstfloat}{htbp}{lop}[section]
\floatname{lstfloat}{Listing}
\def\lstfloatautorefname{Listing} % needed for hyperref/auroref

% Listing style
\definecolor{codegreen}{rgb}{0,0.6,0}
\definecolor{codegray}{rgb}{0.5,0.5,0.5}
\definecolor{codepurple}{rgb}{0.58,0,0.82}
\definecolor{backcolour}{rgb}{0.95,0.95,0.92}

\lstdefinestyle{mystyle}{
    backgroundcolor=\color{backcolour},
    commentstyle=\color{codegreen},
    keywordstyle=\color{blue},
    numberstyle=\tiny\color{codegray},
    stringstyle=\color{codepurple},
    basicstyle=\fontsize{7}{10}\selectfont,
    breakatwhitespace=false,
    breaklines=false,
    captionpos=b,
    keepspaces=true,
    numbers=left,
    numbersep=5pt,
    showspaces=false,
    showstringspaces=false,
    showtabs=false,
    tabsize=2
}

\lstset{style=mystyle}

\makeatletter
\newcommand*{\shifttext}[2]{%
  \settowidth{\@tempdima}{#2}%
  \makebox[\@tempdima]{\hspace*{#1}#2}%
}
\makeatother

\usepackage{pythonhighlight}

%Custom FramedBox Environment
%%Loading 'float' package
\usepackage{float}
%%Customize 'boxed' float style (caption above the body)
\makeatletter
\newcommand\fs@boxedtop
 {\fs@boxed
  \def\@fs@mid{\vspace\abovecaptionskip\relax}%
  \let\@fs@iftopcapt\iftrue
 }
\makeatother
%%Defining float commands
\floatstyle{boxedtop}
\floatname{framedbox}{Box}
\newfloat{framedbox}{hbt}{lob}[section]

% Symbols
%% Differential Upright "d"
\newcommand{\ud}{\,\mathrm{d}}
%% Assemble operator
\DeclareMathOperator*{\assemble}{\text{\Large $ \mathsf{A} $}}
%% Matrices and vectors
\newcommand{\vect}[1]{\bm{#1}}
\newcommand{\mat}[1]{\bm{#1}}
\newcommand{\boldsf}[1]{\boldsymbol{\mathsf{#1}}}

\DeclareMathAlphabet{\pazocal}{OMS}{zplm}{m}{n}

\title{Numerical Methods}
\author{José Luís Passos Vila-Chã}
\date{May 2021}


\begin{document}

\section{Intro}

Hi, in this video, we will try to improve our understanding of the Fourier transform.
The transform is defined like so.

This will be mainly focused on two questions.

First, why does the Fourier transform turns differential equations into algebraic equations?
Here $D$ is differential operator and $f$ some unknown function depending on $t$.
Applying the Fourier transform, we get an algebraic equation, where there is a new unknown function, the Fourier transform of $f$,
$\mathcal F(f)$ function of $w$.
But no differentiation operators are in sight!
We can now solve this algebraic equation and use the inverse Fourier transform to find our initial unknown function $f$.

Second, and closely connected to the first, where does the kernel of the transformation, i.e., $e^{-iwt}$, come from?
How can we interpret it?

\section{Vector/function analogy}

To facilitate our insight, we will use the fact that functions and vectors are very similar in some ways, and this can be rigorously defined.

We can think about a vector, with four components, for example, as a function of sorts.
Its domain is only four numbers, and the value of the function is equal to the corresponding component of the vector.

The internal product between two vectors can be computed like so.
We multiply the corresponding components by each other and sum them all.

To extend this idea, we could consider an increasing number of components.
Even an infinite amount of them, one for each integer, for example.
Taking the next logical step, we can consider instead of a countable number of components, an uncountable number of components.
That is, to each real number, we associate another real number.
But this is precisely the definition of a real function.

Now that we have an uncountable number of components, the previous sum over all the components of the vector used to obtain the interior product must transform into an integral over the domain of the functions.

We can even associate to each component not a real number but a complex one.
The formula for the internal products must be upgraded like this, where the overbar represents the complex conjugate.


\section{Vector explanation}

So now we are going to put this analogy to use.
Remember, we are trying to understand the Fourier transform relative to differential equations.
Let us consider the simplest case.

Since we will start with normal vectors, imagine that the operator \(D\) is not the differentiation operator but some symmetric linear transformation.
We can think about it as a matrix.
It is still an operator, so the conclusions we will reach regarding operators, in general, will still be valid when we look at the differentiation operator.
We also switch to more suggestive symbols for this context.

Looking at this equation, we can think to ourselves, in an ideal world, what would be the slightest effect $D$ could have on a vector.
Perhaps one of the least complicated effects it could have is simply multiplying a vector by a constant.
Obviously, this can be true for every vector, so we ask for this operator $D$, particularly the vectors that don't change direction.
This is the well-known eigenproblem, and the pairs of vectors and constants found are known as the eigenvectors and eigenvalues of the operator.
Because we decided that $D$ is symmetric, the eigenvectors will be orthogonal, and the eigenvalues will be real values.

Now, we can express any vector using the eigenvectors as a basis.
This can be done by projecting the vector using the inner product in each direction-
Summing the contributions in each direction, we get our vector back.
And why is this relevant?

Let's see what happens when we use this on the equation.
Using the fact that $D$ is symmetric, we now have $D v_i$. We know this the same as multiplying $v_i$ by its corresponding eigenvalue.
And just like that, we don't have in our problem any $D$s. The problem has become a system of equations.
So we can find the components of the unknown vector $\bm x$.

\section{Moving on to functions}

So moving on to functions, we will take a look at the Fourier series before looking at the Fourier transform.

We are still considering this simple differential equation.
We asume now that $f$ and $g$ are periodic functions with period $L$.

$D$ is the differential operator, and it is an anti-Hermitian operator.
The dagger is the symbol of the Hermitian conjugate, obtained by transposing and taking the complex conjugate of the operator.
Multiplying $D$ by $-i$, we get a Hermitian operator, which we can think about as the extension of a symmetric operator when going from the real numbers to the complex numbers.

Again looking at this equation, we think to ourselves what would be the smallest effect $D$ could have on a function.
Of course, it could just multiply it by a number.
Writing this in mathematical notation, we have an eigenvalue problem again. The pairs of functions and constants found are known as the eigenfunctions and eigenvalues.
We use $-iD$ because of its Hermicity so that the eigenfunctions are orthogonal to each other.
The eigenfunctions of the operator $-iD$ are $e^{i\lambda t}/\sqrt{2\pi}$.
We can choose from these functions for our basis, whose period is an integer multiple of $L$.
These form an orthonormal basis for the periodic functions of period $L$.

Now, we can express any such function using this basis.
This is precisely the Fourier series of the function.

Let us see what happens when using this representation on the equation.
We assume that all relevant functions have converging Fourier series.
Using the fact that $D$ is anti-Hermitian, we now have $D e^{i\lambda t}$, and we know this the same as multiplying $e^{i\lambda t}$ by $i\lambda$.
And just like that, there is no $D$ insight. The problem has become a system of equations.
An infinite number of them.
So we can find the number of coefficients we wish and approximate the unknown function through the truncated Fourier series.

\section{Fourier transform}

Finally, we are going to look at the Fourier transform.

We are still considering this simple differential equation.
$D$ is still the differential operator with the same properties.
We consider the eigenproblem again and for the same reasons.

Now for our basis, we consider all the eigenfunctions where $\lambda$ is real.
Now, we can express any appropriate function using this basis.

This is the so-called Fourier integral, where the interior product inside the integral is the Fourier transform.

Substituting above and using the same argument as before, we can transform the differential equation into an algebraic equation.
Having found the Fourier transform, we can get back the unknown function through the Fourier integral.

\section{Outro}

Answering our initial questions, we have seen that the Fourier transform transforms differential equations into algebraic equations. It represents the function using a basis of functions where the differentiation operator is no more than multiplying the function by a number.
This basis of functions is the $e^{iwt}$ this is why its complex conjugate is the kernel of the Fourier transform.







\end{document}
